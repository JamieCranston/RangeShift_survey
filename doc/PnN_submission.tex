% Options for packages loaded elsewhere
\PassOptionsToPackage{unicode}{hyperref}
\PassOptionsToPackage{hyphens}{url}
%
\documentclass[
]{article}
\usepackage{amsmath,amssymb}
\usepackage{lmodern}
\usepackage{ifxetex,ifluatex}
\ifnum 0\ifxetex 1\fi\ifluatex 1\fi=0 % if pdftex
  \usepackage[T1]{fontenc}
  \usepackage[utf8]{inputenc}
  \usepackage{textcomp} % provide euro and other symbols
\else % if luatex or xetex
  \usepackage{unicode-math}
  \defaultfontfeatures{Scale=MatchLowercase}
  \defaultfontfeatures[\rmfamily]{Ligatures=TeX,Scale=1}
\fi
% Use upquote if available, for straight quotes in verbatim environments
\IfFileExists{upquote.sty}{\usepackage{upquote}}{}
\IfFileExists{microtype.sty}{% use microtype if available
  \usepackage[]{microtype}
  \UseMicrotypeSet[protrusion]{basicmath} % disable protrusion for tt fonts
}{}
\makeatletter
\@ifundefined{KOMAClassName}{% if non-KOMA class
  \IfFileExists{parskip.sty}{%
    \usepackage{parskip}
  }{% else
    \setlength{\parindent}{0pt}
    \setlength{\parskip}{6pt plus 2pt minus 1pt}}
}{% if KOMA class
  \KOMAoptions{parskip=half}}
\makeatother
\usepackage{xcolor}
\IfFileExists{xurl.sty}{\usepackage{xurl}}{} % add URL line breaks if available
\IfFileExists{bookmark.sty}{\usepackage{bookmark}}{\usepackage{hyperref}}
\hypersetup{
  pdftitle={UK wildlife recorders cautiously welcome range-shifting species but incline against intervention to promote or control their establishment},
  pdfauthor={James Cranston; Sarah L. Crowley; Regan Early},
  hidelinks,
  pdfcreator={LaTeX via pandoc}}
\urlstyle{same} % disable monospaced font for URLs
\usepackage[margin=1in]{geometry}
\usepackage{longtable,booktabs,array}
\usepackage{calc} % for calculating minipage widths
% Correct order of tables after \paragraph or \subparagraph
\usepackage{etoolbox}
\makeatletter
\patchcmd\longtable{\par}{\if@noskipsec\mbox{}\fi\par}{}{}
\makeatother
% Allow footnotes in longtable head/foot
\IfFileExists{footnotehyper.sty}{\usepackage{footnotehyper}}{\usepackage{footnote}}
\makesavenoteenv{longtable}
\usepackage{graphicx}
\makeatletter
\def\maxwidth{\ifdim\Gin@nat@width>\linewidth\linewidth\else\Gin@nat@width\fi}
\def\maxheight{\ifdim\Gin@nat@height>\textheight\textheight\else\Gin@nat@height\fi}
\makeatother
% Scale images if necessary, so that they will not overflow the page
% margins by default, and it is still possible to overwrite the defaults
% using explicit options in \includegraphics[width, height, ...]{}
\setkeys{Gin}{width=\maxwidth,height=\maxheight,keepaspectratio}
% Set default figure placement to htbp
\makeatletter
\def\fps@figure{htbp}
\makeatother
\setlength{\emergencystretch}{3em} % prevent overfull lines
\providecommand{\tightlist}{%
  \setlength{\itemsep}{0pt}\setlength{\parskip}{0pt}}
\setcounter{secnumdepth}{-\maxdimen} % remove section numbering
\ifluatex
  \usepackage{selnolig}  % disable illegal ligatures
\fi

\title{UK wildlife recorders cautiously welcome range-shifting species
but incline against intervention to promote or control their
establishment}
\author{James Cranston\footnote{Centre for Ecology and Conservation,
  College of Life and Environmental Sciences, University of Exeter,
  Penryn, Cornwall, TR10 9FE, UK ,
  \href{mailto:j.cranston@protonmail.com}{\nolinkurl{j.cranston@protonmail.com}}} \and Sarah
L. Crowley\footnote{Centre for Geography and Environmental Science,
  College of Life and Environmental Sciences, University of Exeter,
  Penryn, Cornwall, TR10 9FE, UK.} \and Regan Early\footnote{Centre for
  Ecology and Conservation, College of Life and Environmental Sciences,
  University of Exeter, Penryn, Cornwall, TR10 9FE, UK ,
  \href{mailto:r.early@exeter.ac.uk}{\nolinkurl{r.early@exeter.ac.uk}}.}}
\date{25/07/2021}

\begin{document}
\maketitle

\hypertarget{abstract}{%
\subsection{Abstract:}\label{abstract}}

\begin{enumerate}
\def\labelenumi{\arabic{enumi}.}
\tightlist
\item
  The global redistribution of species due to climate change and other
  anthropogenic causes is driving novel human-wildlife interactions with
  complex consequences. On the one hand, range- shifting species could
  disrupt recipient ecosystems. On the other, these species may be
  contracting in their historic range, contributing to loss of
  biodiversity there. Given that arriving range-shifting species could
  also perhaps have positive effects on recipient ecosystems, there is
  {[}in principle{]} a net benefit equation to be calculated. Thus,
  public opinion on these species may be divided and they may present a
  unique challenge to wildlife management.\\
\item
  We surveyed the opinion of wildlife recorders about the establishment
  and management of eight birds and eight insects whose ranges have
  recently shifted into the United Kingdom. We asked whether
  respondents' attitudes were explained by the species' or respondents'
  characteristics, and whether or not climate change was emphasised as a
  cause of range-shift. We also conducted qualitative analysis of the
  recorders' text responses to contextualise these results.\\
\item
  Attitudes to range-shifting species were mostly positive but were more
  ambivalent for less familiar taxa and for insects compared with birds.
  Respondents were strongly opposed to eradicating or controlling new
  range-shifters, and to management aimed to increase their numbers.
  Whether climate change was presented as the cause of range-shifts did
  not affect attitudes, likely because respondents assumed climate
  change was the driver regardless.\\
\item
  These findings suggest that it will be difficult to generate support
  for active management to support or hinder species' redistribution,
  particularly for invertebrate or overlooked species amongst wildlife
  recorders. However, the positive attitudes suggest that on the whole
  range-shifting species are viewed sympathetically. Engaging with
  wildlife recorders may represent an opportunity to garner support for
  conservation actions which will benefit both currently native and
  arriving species, such as improvements to habitat quality and
  connectivity.
\end{enumerate}

\textbf{Keywords:} Attitudes, climate change, citizen science,
human--wildlife interaction, public opinion, range-shift, species'
redistribution, wildlife management

\hypertarget{introduction}{%
\subsection{Introduction:}\label{introduction}}

Many species' distributions are shifting rapidly in the 21st century as
they track climate and habitat change (Pecl et al 2017). The number and
abundance of these `range-shifting' species moving across national
boundaries is rapidly increasing (Chen et al 2011; Gurney et al 2015,
Latombe et al 2020) and studies of their effects on current ecosystems
are still few in number. Range expansions into new regions, while being
crucial to species adaptation to environmental change, also have the
potential to alter and perturb existing ecosystems (Wallingford et al,
2020). There is therefore a pressing need to explicitly consider the
role of range-shifting species in conservation and wildlife management.
Management implications will vary for different ecosystems, societies,
and stakeholders (Tebboth, Few, Assen, \& Degefu, 2020) in part due to
different ecological conditions but also because of differing public
reactions to range-shifting species -- the focus of this paper. Species
perceived as harmful may encounter negative reactions, while those
viewed as threatened may receive more positive ones.

The patterns of movement for these range-shifting species which are
moving without human assistance may be complex. Some species may only
change their distributions within their historic range while others may
show expansions or retractions at range edges (Lenoir and Svenning,
2015). Differing patterns could affect public reactions. Here, we focus
solely on those species which are contracting or static in their
distribution in historic regions but are expanding into new regions
(here after range-shifting species) and the implications of these
arrivals for their new regions. Climate change's role in shaping these
perceptions is unknown despite its being a well-established driver of
range-shift. On the one hand, climate change's association with newly
arriving species could tarnish them as dangerous `climate opportunists'
as climate change is considered a serious threat to humanity and mostly
harmful to biodiversity (Newbold, 2018). On the other hand, climate
change is anthropogenic in origin, therefore people could view arriving
species as `ecological refugees', which we have a moral responsibility
to protect (Urban, 2020).

The importance of public attitudes in shaping ecological outcomes has
been demonstrated by research into biological invasions (Andreu, Vilà,
\& Hulme, 2009), species reintroductions (Klich, Olech, Łopucki, \&
Danik, 2018) and human-wildlife interactions (Baruch-Mordo, Breck,
Wilson, \& Broderick, 2009). A compelling example of how public
opposition can hinder conservation efforts was the Scottish government's
campaign to lethally eradicate non-native hedgehogs from South Uist
(after a deliberate introduction in 1974), which subsequently provoked a
reactive coalition made up of opposed NGOs -- the ``Uist Hedgehog
Rescue'' (Crowley, Hinchliffe, \& McDonald, 2017). A better
understanding of what shapes people's attitudes to species may help
inform conservation strategies. For example, species that are viewed as
charismatic could act as flagship species (Ma et al., 2016), leveraging
public support for conservation of associated range-shifting species.
Alternatively, species that are viewed negatively which may be more
challenging to conserve, for example wasps (Sumner, Law, \& Cini, 2018).
Other species may pass beneath the public's notice altogether, which
could avoid concern, but could also make it more difficult to enact
measures that promote or restrict their establishment. Of course,
attitudes can vary greatly between human communities, and shift rapidly
(Jones, et al.~2020). Therefore, establishing baselines in the attitudes
and awareness of different stake-holder groups may help to inform future
management strategies. For example, knowledge of negative attitudes
could suggest initial education programs (Bath, 1989). And evidence of
hardening attitudes might suggest a backlash against a particular
management strategy and a possible need for review.

Investigating attitudes to range-shifters is timely (Naujokaitis-Lewis,
Pomara, \& Zuckerberg, 2018). The rate of climate change continues to
accelerate, and it is uncertain how many species will be able to shift
quickly enough to track their climatic niche across increasingly
human-dominated landscapes (Schloss et al., 2012). This has led some
scientists to advocate new and bold approaches, such as assisted
translocation (Lunt et al., 2013). For species which require large-scale
interventions, public attitudes are likely to be important, particularly
in densely populated areas or where there is potential conflict
(O'Rourke, 2014). In either scenario, the evidence strongly suggests
that management is more effective when stakeholders are successfully
engaged (Crowley et al., 2017; Redpath et al., 2013). Apart from the
evidence for its efficacy, stakeholder engagement is a normative
concern. Democratic governance relies on accountability to citizens and
public opinion therefore forms an important input into legitimising
decisions (Berry, Koski, Verkuijl, Strambo, \& Piggot, 2019; Kiss,
2014).

Wildlife recorders are a key group with whom to engage when considering
range-shifting species. We defined wildlife recorders as volunteers
contributing to datasets of the times and locations of species
occurrences, often as part of a local or national scheme. Recorders are
often the first to both identify and report invasive species and also
note the arrival of range-shifters (Brown et al., 2018). In addition,
they provide much of the raw data underpinning conservation decisions in
the UK (Pocock, Roy, Preston, \& Roy, 2015). As such, wildlife recorders
are a group likely to have greater awareness of range-shifts than the
wider population, meaning that their attitudes may be more developed and
better informed. Furthermore, recorders are interesting in their own
right, as their attitudes could indicate their willingness to adapt
their recording to better inform decision making on range-shift
management. The UK provides a useful case-study, as it has a
well-documented fauna and a very active volunteer recording community.
There have also been a considerable number of arriving range-shifters
over recent years (Gurney, 2015).

Many factors, aside from climate change, might influence wildlife
recorders' attitudes to range-shifters. There is considerable evidence
that taxonomic group has a strong effect on public attitudes, and there
is a growing body of theory covering possible mechanisms (Troudet,
Grandcolas, Blin, Vignes-Lebbe, \& Legendre, 2017). Furthermore,
recorders are a heterogeneous group (Dawson \& Martin, 2015).
Individuals' attitudes might also differ based on their personal
attributes, including level of knowledge about range-shift or their
views on the relationship between humans and nature (Sharp, Larson, \&
Green, 2011).

Our study explores how range-shifting species are viewed by wildlife
recorders in the UK through an online survey. Specifically, we sought to
learn how aware recorders were of different range-shifting species and
whether they viewed range-shifters predominantly positively or
negatively. We asked the extent to which the identity of the
range-shifting species, and the attributes of respondents, affect
attitudes. Finally, we asked what attitudes recorders expressed towards
potential management of range-shifters / new arrivals, including both
positive management aimed to help them establish and spread, and
negative management to control or prevent arrival. This was coupled with
additional thematic analysis of respondents' written answers that
explained their attitudes to the species and their management as well as
how they perceived climate change as affecting those attitudes. If
attitudes are positive, then future management may be drawn towards
assisting range-shifting species and it may be harder to protect
recipient ecosystems from any that are harmful. On the other hand,
negative attitudes could drive management to make it harder for
threatened species to shift their ranges.

\hypertarget{materials-methods}{%
\subsection{Materials \& Methods:}\label{materials-methods}}

\hypertarget{survey-participant-selection}{%
\subsubsection{Survey participant
selection}\label{survey-participant-selection}}

UK wildlife recorders' perspectives on range-shifting species were
collected using the online survey software LimeSurvey
(\url{http://www.limesurvey.org}). A targeted sampling strategy was used
to maximise the response rate from our population of interest and the
survey was open for responses between 15th April 2019 and 1st June 2019.
A link to the survey was distributed to recorders using selected
Facebook groups. These groups were identified using two methods.
Initially, all schemes from the Biological Records Centre and British
Trust for Ornithology affiliated bird clubs which had a detectable
presence on Facebook were contacted. Secondly, Facebook groups were
searched for using the following terms: ``Field'', ``Natural History'',
``Naturalist'' and ``Record*''. A link to the survey was also posted on
the National Biodiversity Network website and in the National Forum for
Biological Recorders newsletter. The survey design and administration
for this study was approved by the College of Life and Environmental
Sciences' Ethics Committee (Penryn) at the University of Exeter,
reference eCORN000039. We ensured that respondent's informed consent was
obtained before they participated. No compulsory questions were included
to avoid steering respondents to answer questions on which they had no
opinion and minimise the effects of survey fatigue. A full copy of the
questionnaire is available in Appendix 1, including the welcome page
where respondents were provided with contact details for the lead
researcher, the purpose of the study and a check box to indicate their
consent to participate.

\hypertarget{survey-design}{%
\subsubsection{Survey design}\label{survey-design}}

Recording behaviour, level of knowledge and relationship with nature
Three questions were asked to characterise respondents' recording
behaviour. First, respondents were asked which taxonomic groups they
recorded from a checklist. Two questions then assessed the respondents'
level of engagement: respondents were asked how long they had been
sharing or submitting wildlife sightings or records, then respondents
identified which recording activities they performed. This was treated
as an ordered factor with four options: sharing wildlife sightings
informally (e.g.~via Facebook) (1 - least engaged), submitting sightings
as Records for a scheme (2), verifying biological records (3), and
organising a recording scheme (4 - most engaged). The maximum engagement
level of each respondent was recorded.

Level of knowledge was analysed similarly to Verbrugge et al (2013):
respondents were asked whether they had heard of any species
establishing in the UK having arrived under their own powers of
dispersal. It was clarified that this did not include human-introduced
species. If confident, respondents were asked to name a naturally
establishing species. Three response levels were recorded: no (0); yes,
but couldn't name a correct example (1); yes, and named a correct
example (2).

We characterised the respondent's relationship with nature using a
shortened survey from (Verbrugge et al., 2013). This consisted of twelve
Likert-type questions, with three testing each of four theoretical modes
of relationship: master (humans stand above nature and can exploit if
for their needs), steward (humans stand above nature but have a
responsibility to preserve it), partner (humans and nature are separate
entities which should work equally together to develop), and participant
(humans are both biologically and spiritually part of nature, no
dualistic ontology) outlined in de Groot et al.(2011) and de Groot \&
van den Born (2003). We recorded the participant's mean score for each
set of three questions evaluating their manner of human-nature
relationship. We interpreted the mode with the highest score as
indicating the strongest alignment.

Respondents' age, gender, level of education and postcode were collected
to contextualise the results and to help control sources of potential
unknown variation. Employment in the wildlife sector was also included
as it has been demonstrated to affect attitudes to species management in
the literature on invasive and pest species (Bremner \& Park, 2007).

\hypertarget{climate-change-experimental-approach}{%
\paragraph{Climate change experimental
approach}\label{climate-change-experimental-approach}}

In our experimental approach we either presented an image of a road sign
against a neutral background displaying the term `climate change' or a
control image identical but displaying the word `information'. Each
image was accompanied with a brief explanation that new species were
establishing in the UK, either referencing climate change (experimental
treatment) or not (control). Respondents were then asked to think about
climate change (experimental treatment) or species range-shift (control)
and write down the first word(s) or phrases which came to mind (Appendix
1). Later in the survey we asked respondents whether they thought that
the arrival of species due to climate change had distinct management
implications, compared to other drivers of range-shift.

\hypertarget{attitudes-to-species-and-their-management}{%
\paragraph{Attitudes to species and their
management}\label{attitudes-to-species-and-their-management}}

Each respondent was shown four species vignettes: two vertebrate and two
invertebrate range-shifters selected at random from a pool of 16 animals
(Table S1). These species were chosen to represent a broad taxonomic
range, were all recent arrivals (\textless30 years) and had English
common names and photos available. Vignettes were presented in a random
order to avoid order effects (Auspurg \& Jäckle, 2017). The vignette
consisted of a header repeating the information shown previously in the
climate change experimental treatment or control as appropriate,
followed by an image of the species obtained from Flickr. We attempted
to choose neutral images, where the subjects were at rest, in centre
frame against natural backgrounds, similarly to Borgi and Cirulli
(2015). We also presented the common and scientific species names, a
written description of its appearance, its habitat preference and
average body length in centimetres. As information on impacts and
distribution were not equally available across our chosen species we did
not incorporate information on these aspects in the vignettes.

For each species, we asked respondents whether they had seen the species
in the UK. They were then asked how they felt about the species
establishing in the UK on a response scale of seven options from very
negative (1) to very positive (7). Respondents were then presented with
five different management actions for each species (Table 1), about
which they rated their feelings along the same scale. Respondents could
also respond to some open text questions on the attitude and management
questions, which we used for qualitative analysis.

\begin{longtable}[]{@{}
  >{\raggedright\arraybackslash}p{(\columnwidth - 0\tabcolsep) * \real{1.00}}@{}}
\caption{Management options for species presented to respondents. The
common species name from the vignette was used in place of the
``Species'' placeholder.}\tabularnewline
\toprule
Management.Options \\
\midrule
\endfirsthead
\toprule
Management.Options \\
\midrule
\endhead
Remove -- management should actively try to reduce ``Species X''
populations and if feasible remove them. \\
Mitigate - management should try to decrease ``Species X'' populations
where possible and control them if not. \\
Non-Intervention -- management of ``Species X'' be avoided where
possible and minimal where not. \\
Adapt -- management should try to increase ``Species X'' populations
where possible and conserve them if not. \\
Support -- management should actively try to increase ``Species X''
populations and if feasible introduce them. \\
\bottomrule
\end{longtable}

\hypertarget{statistical-methods}{%
\subsubsection{Statistical methods}\label{statistical-methods}}

We constructed a multinomial logit model in R (R Core Team, 2020) to
describe respondent attitudes to arriving range-shifters and infer which
factors affected those attitudes. Answers to questions regarding
attitudes were collapsed into three categories for each response:
positive (original response = 5 (a bit positive), 6 or 7 (very
positive)), negative (original response = 1 (very negative) , 2 or 3 (a
bit negative) ), or neutral (original response = 4 (neutral)). We
modelled these three categories using a Bayesian multinomial model
(Appendix 2a) using the R package BRMS (Bürkner, 2018). We investigated
the fixed effects of respondent gender, education, age, years recording,
level of knowledge, employment in the wildlife sector, engagement with
recording, the climate change experimental treatment and whether they
had seen the species or if it was part of a group they recorded. We also
included species and respondent as random effects.

To help regularise the model, all fixed effects were estimated using a
horseshoe prior. Following (Piironen \& Vehtari, 2017) we determined the
global scale parameter from an a priori assumption of the expected ratio
of zero to non-zero coefficients. We chose 0.33. The prior for the
standard deviation of both random effects was weakly informative
(Student's t, df = 3, mean = 0, scale = 2.5). Model convergence was
assessed using visual examination of trace plots and the Gelman-Rubin
diagnostics (Brooks \& Gelman, 1998), which for all parameters was under
1.05. Bulk and tail estimated sample sizes were \textgreater1000 for all
parameters. We followed the same process to model respondents' attitudes
to the five different management options as a multivariate model
(Appendix 2b), but we also included attitude to the species as a fixed
effect. We report the posterior mean and lower (LCI) and upper (UCI)
95\% credible intervals for all model parameters stated in the results.
Posterior summaries for all parameters for both models are available in
Appendix 3.

\hypertarget{assessing-the-rationale-underlying-attitudes-through-thematic-coding}{%
\subsubsection{Assessing the rationale underlying attitudes through
thematic
coding}\label{assessing-the-rationale-underlying-attitudes-through-thematic-coding}}

We explored the written answers accompanying each quantitative question
on attitudes in order to identify `themes' in the underlying rationales
which might explain respondents' attitudinal responses to
range-shifters. Coding was carried out in NVivo 12 (QSR International,
1999) using an inductive approach to create a novel framework to
describe the responses. In order to tie themes to attitudes, we had to
adopt two approaches. For the question on attitudes to species we simply
calculated, by theme or subtheme, the proportion of reference coded that
came from a respondent with a positive, neutral, or negative attitude to
that species. However, for the question on attitudes to species
management, we had to classify respondents' attitudes into clusters due
to the multidimensional nature of the question (5 management aspects),
using Multiple Correspondence Analysis (MCA) (Rouanet \& Le Roux, 2010).
We then plotted the proportion of references coded that came from a
respondent with a given attitude (1st question) or assigned cluster (2nd
question) for each theme or sub-theme.

\hypertarget{results}{%
\subsection{Results:}\label{results}}

\hypertarget{respondent-characteristics}{%
\subsubsection{Respondent
characteristics}\label{respondent-characteristics}}

In total, 506 respondents clicked on the survey link and 315 continued
to survey completion (median time to complete 21 mins). The respondents
had a median age of 56 (Q1 = 44; Q3 = 63), older than the UK median
(39), were significantly male biased compared to the 2011 UK population
Census (63.5\% male, Figure S1), and had attained higher levels of
educational qualification than expected relative to the 2011 census
(Figure S2). We found 98\% of our respondents were aware of
range-shifting species before taking the survey, with most able to name
at least one recently arrived species (Figure S3). Respondents
represented a spectrum of involvement with UK recording (Median years
recording = 10, Q1 = 5, Q3 = 25, Figure S4). 40.6\% of respondents
self-identified as working in the wildlife sector. Respondents most
strongly aligned with a `stewardship' vision of nature (Stewardship =
241, Participant = 9, Partner = 3, Master = 1, Tied scores = 61).
Respondents were distributed across the whole of the UK (Figure S5).

\hypertarget{wildlife-recorders-attitudes-to-range-shifting-species}{%
\subsubsection{Wildlife recorders attitudes to range-shifting
species}\label{wildlife-recorders-attitudes-to-range-shifting-species}}

Respondents held positive attitudes to range-shifting species, with
60.2\% being `a bit positive' or more, 35.6\% neutral, and only 4.2\% `a
bit negative' or more (Figure 1). Results broken down by species and
taxonomic groups showed that bird species were viewed most positively,
followed by dragonflies; the shield bugs and the wasp (D. saxonica) were
viewed least positively. However, even for D. saxonica the majority of
people held a neutral rather than negative attitude (Figure S6).

\includegraphics{D:/Academic_Work/PhD Thesis/Chapter_2/Review Round 2/P\&N/doc/PnN_submission_files/figure-latex/fig_1-1.pdf}

Figure 1 - Respondent attitudes towards the establishment of new
range-shifters in the UK. Positive attitudes are shown in blue, neutral
in grey and negative in red. All responses are summarised at the top,
responses below are split by those who had seen the range-shifter or not
and those who did or did not habitually record that species' taxonomic
group.

\hypertarget{factors-important-to-attitudes-on-range-shifting-species}{%
\subsubsection{Factors important to attitudes on range-shifting
species}\label{factors-important-to-attitudes-on-range-shifting-species}}

The climate change treatment did not appear to effect respondents'
attitudes to range-shifting species and respondent's frequently
mentioned climate change when prompted for words they associated with
range-shift (Appendix 4). Of our explanatory variables only two were
effective predictors of attitudes to range-shifting species. We found
that when respondents had seen the species that they were being asked
about (Parameter log-odds positive vs negative response mean = 1.76, LCI
= 0.34, UCI = 2.51) or when it was part of a group they were involved in
recording (Parameter log-odds positive vs negative response mean = 1.45,
LCI = 0.87, UCI = 2.06) they were more likely to have a positive
attitude towards it (Figure 2a). All other fixed effects (see Section
2.3 -- Statistical Methods) in the converged model were small (95\%
credible intervals overlapped 0). However, the effect of respondent
(Std. Dev. log-odds positive vs negative response: mean = 2.52, LCI
=1.95, UCI = 3.15) and species (Std. Dev. log-odds positive vs negative
response mean = 1.53, LCI = 0.96, UCI = 2.40) were large relative to the
effect of whether the species was in a recorded group or whether it had
been seen, Figure 2(b). All of our species groups were recorded by at
least a third of recorders in our sample (Figure S7).

\includegraphics{D:/Academic_Work/PhD Thesis/Chapter_2/Review Round 2/P\&N/doc/PnN_submission_files/figure-latex/fig_2-1.pdf}

Figure 2(a) The predicted probability of a respondent having a positive
(blue), neutral (grey) or negative (red) attitude to a species depending
on whether they had seen the species and whether it was in a taxonomic
group they recorded. The point shows the median of the posterior
probability and the bars the 95\% credible intervals. 2(b) The predicted
probability of a respondent having a positive attitude for a given
species (coloured by taxonomic group). \#\# Attitudes to management
intervention either to promote or restrict range-shifting species\\
Wildlife recorders most favoured a non-interventionist approach, which
was the only option with more positive attitudes (56.2\%) than negative
(8.6\%) (Figure 3 and Figure S8). Adapting existing ecosystems to cope
with range-shifters (P(Pro), mean = 0.26 LCI = 0.24, UCI= 0.29) was
viewed more favourably than supporting range-shifters (P(Pro), mean =
0.13, LCI = 0.11, UCI= 0.15). There was less opposition to supporting
range-shifters (P(Anti), LCI = 0.48, UCI = 0.53) than there was to
controlling (P(Anti), mean = 0.71, LCI = 0.69, UCI= 0.74) or removing
them (P(Anti), mean = 0.74, LCI = 0.72, UCI= 0.76). There was no
difference in approval between mitigation and removal. Attitudes to
management options for any given species were strongly related to the
attitudes respondents held to the species themselves (Figure S9).

\includegraphics{D:/Academic_Work/PhD Thesis/Chapter_2/Review Round 2/P\&N/doc/PnN_submission_files/figure-latex/fig_3-1.pdf}

Figure 3 -- Respondent attitudes to the five management options
presented in full in Table 1, Attitudes are colour coded from very
negative (dark pink) through neutral (grey) to very positive (dark
blue). MCA found four clusters in the quantitative responses to
questions about range-shifters' management, and we interpret them as
loosely representing four putative attitude groups: range-shifter
supporters, non-interventionists, neutrals, and range-shifter wary
(Figure S10).

\hypertarget{emerging-themes-from-qualitative-analysis-surrounding-attitudes-to-range-shift}{%
\subsubsection{Emerging themes from qualitative analysis surrounding
attitudes to
range-shift}\label{emerging-themes-from-qualitative-analysis-surrounding-attitudes-to-range-shift}}

Our thematic coding of the free text responses highlighted several
recurring themes in our respondents' answers. We identified four themes
which cut across both respondents' explanations for their attitudes to
the species themselves (Table S2) and their attitudes to species
management (Table 2). However, the prevalence of each theme differed,
and different sub-themes were identified for attitudes to species
themselves and their management. With respect to attitudes to species,
the most common theme was the potential direct costs and benefits of the
arriving species, both socio-economic and ecological. There was a strong
emphasis on social benefits here, often related to personal experiences.
The second most recurrent theme was that of generalised principles about
human intervention in nature and whether these range-shifters were
arriving ``naturally''. Some respondents noted a mixed feeling: they
felt positive about the species' establishment but worried about its
perceived anthropogenic driver -- climate change. The third theme was
the respondents' perceived feasibility of managing the range-shifters
and only appeared infrequently for this question. The final theme was
the idea of nativeness, where some respondents argued that native
species should be prioritised over range-shifters.

With respect to species management, the most common theme arising was
the costs and benefits of the species in question. However, unlike
responses to species themselves, respondents raised the potential costs
of range-shifting species and most stated that they would consider
management if range-shifters caused negative impacts. Those who thought
range-shifters could have a positive impact were more likely to be
range-shift supporters, while those who thought that a negative impact
was possible were more likely to be neutral (Table 2). The theme around
human intervention in nature was expressed more commonly in relation to
management than in relation to species themselves, revolving around
ideas of letting a range-shifter develop its own path naturally without
management. Animal rights emerged as a minor subtheme within this theme.
The theme of costs and benefits of management action received
substantial attention, with considerable scepticism of the potential
efficacy of management. On the theme of nativeness, there was a majority
view that management should protect native species over range-shifters.
However, a minority argued that all arriving range-shifters should
effectively be viewed as native and therefore any negative impacts
should be managed similarly to impacts from resident problematic
species.

With respect to whether respondents thought that a causal effect of
climate change on species range-shifts had implications for management
we found a range of perspectives (Table S3). Some respondents thought
that climate-driven range-shifters deserved specific attention as they
might be losing range elsewhere, would be important for future climate
adaptation, and due to a moral responsibility arising from humanity's
culpability for climate change as ``they are being pushed out of their
range, and it's our fault!''. A similar number of respondents thought
that management to restrict climate-driven range-shifters was futile,
arguing that the focus should be on the climate change ``cause'' not the
species ``symptoms''; preventing the arrival of even harmful species was
too difficult and, even though climate change was anthropogenic, the
species dispersal response to it was natural and should be accepted:

``I think that this is a paradox as global warming is a direct result of
human impact yet in species colonisation context{[}s{]} humans should
stay out of it.''

Finally, around a third of references coded suggested that climate
change was not the most important factor in forming a management
response to these species. Instead, these participants thought that the
impact of range-shifting species on the recipient ecosystem should shape
the response, or that people's responsibility is to protect biodiversity
as a whole rather than to focus on specific cases.

Table 2 -- Our coding framework's themes applied to written responses
about range-shifting species management, each illustrated by a quote. RC
indicates the number of references coded. The bar chart accompanying
each theme and subtheme shows the percentage of references coded that
were from each group of respondents within each theme (support
range-shifters = blue, non-intervention = yellow; neutral = grey, wary
of range-shifters = red. Dashed lines show intervals of 25\%). Theme
Sub-theme Summary Example Quote Cost / Benefit ratio from arrival of
range-shifting species (RC = 539) Positive effects on recipient
ecosystems (RC = 51) Range-shifters will affect the recipient ecosystem
positively, via both ecological (e.g.~increased resilience) and social
(e.g.~eco-tourism) mechanisms. ``Climate change may mean native species
can no longer tolerate conditions so better to have a replacement
species rather than none at all'' Minimal effects on recipient
ecosystems (RC = 150) Range-shifters will have minimal effects, either
positive or negative, on the recipient ecosystem. ``There is no reason
to control this species since it is not harmful.'' Negative effects on
recipient ecosystems (RC = 273) Range-shifters are perceived as having
potentially negative effects on the recipient ecosystem, with a focus on
risk rather than proven effects. ``As a non native species with
potential to become invasive (I would assume) they shouldn't be actively
encouraged to establish. It's challenging to control populations and
would be costly and resource heavy to do so to the extent of
extermination'' Conservation status of the range-shifter (RC = 65)
Threatened range-shifters deserve conservation assistance, actions to
promote the global conservation status of a species are important.
``Species specific actions should only be prioritised against species
that are threatened.'' Human intervention in nature (RC = 274) Animals'
rights (RC = 14) Management must respect the rights of sentient
creatures to exist. ``Every living thing has the right not to be
persecuted.'' Allowing nature to take its course is preferable to human
intervention. (RC = 260) Allowing natural processes to shape
range-shifts will result in better outcomes than trying to manage them
directly. ``Let things alone and Mother Nature will look after itself
without any interference'' Cost / Benefit ratio of management actions
(RC = 197) Management to control range-shifters will have negative
ecological costs (RC = 27) Management to control range-shifters will
have associated negative effects on the recipient ecosystem,
e.g.~insecticide use. ``It seems likely that reduction by `management'
would have unforeseen undesirable implications for other species.''
Management to help range-shifters will have additional positive benefits
(RC = 23) Management to help range-shifting species will have additional
positive effects on the recipient ecosystem, e.g.~habitat improvement.
``Providing habitat for these colonisers often provides much needed
habitat for other struggling UK species.'' Management to assist
range-shifters gives no additional benefit (RC = 90) Management to
assist range-shifters establish will not convey any additional benefit
to the chances of establishment success. ``There is no need to actively
conserve / increase the species as it is spreading naturally and numbers
are rising.'' Management to reduce risk from range-shifters is too
difficult (RC = 57) Management attempts to control range-shifters will
be very expensive and ineffective, some may be politically or
practically infeasible. ``I suspect it's pretty impossible to do
anything to stop (or help) this species.'' Nativeness (RC = 57) Natives
should be prioritised for support (RC = 46) Native species should be our
priority. We should conserve native species rather than support
range-shifters and protect native species in favour of range-shifters s.
``I think we should be focusing our conservation efforts on our native
species first'' Range-shifters should be managed as if they were native
(RC = 11) There should be no difference in our treatment of
range-shifters or native species. The arrival of range-shifters is
``natural'' so they should be considered equal. ``As they arrived of
their own accord might as well be considered native and managed as
native wildlife''

\hypertarget{discussion}{%
\subsection{Discussion}\label{discussion}}

This study aimed to discover how range-shifting species are viewed by a
key stakeholder group, and consider how far these perspectives might
reflect public opinion and impact wildlife management. We found that
awareness of the presence of range-shifting species was high. Moreover,
most recorders had positive attitudes towards these species
establishing. The species in question and recorder familiarity with the
species both predicted recorders' attitudes to the establishment.
However, this positivity did not manifest as favouring active forms of
management to assist establishment. Instead, most favoured
non-intervention in the range-shift process. Recorders were also
strongly against efforts to hinder range-shift. These positive responses
indicate that wildlife recorders value range-shifters. The sources of
this value emerged in the qualitative comments. Many respondents talked
about personal experiences with the species, for example ``fabulous
bird, what a joy to see them'', others about ecosystem services
(principally pollination) and reduced extinction risk (Table S2). This
value suggests that people perceive socio-ecological benefits from the
arrival of range-shifters.

The variation that we found amongst recorders' attitudes towards
different species suggests that there will be winners and losers in the
battle for positive public reception. This finding supports the existing
literature on taxonomic biases which finds that invertebrates are often
perceived more negatively than vertebrates when considering
reintroductions (Seddon, Soorae, \& Launay, 2005) and invasions (Bremner
\& Park, 2007). This effect is lessened for aesthetically attractive
species like dragonflies and butterflies (Shipley \& Bixler, 2017), as
we found. In fact, it is perhaps surprising that attitudes were mostly
neutral rather than negative for less aesthetic invertebrates. This
probably reflects recorders' desire for further information on which to
base their judgements. Their opinions were often balanced, for instance
that some scary or unpopular species such as wasps also provided
important ecosystem services such as pest control. This more reserved
stance may not be shared by less informed groups. Species charisma has
been shown to be very influential in both the management and spread of
invasive species (Jaric et al., 2020) and our results suggest that this
may also be the case for range-shifters. Further research could refine
what attributes that people are responding to distinguish better and
less popular species to better inform management decisions. For
instance, by highlighting which harmful species would represent a risk
of public resentment against control efforts or, conversely, which
species might be used to attract funding or public engagement.

Our results may not generalise to less ecologically informed publics or
other demographic groups (Figure S4). In the future, it would be useful
to compare these results with those of other stakeholders, such as
landowners, land-managers, and scientists to better understand potential
differences. The UK is an outlier compared to most countries in its
human population density and GDP. Evidence suggests that more
established economies have greater environmental concern (Franzen \&
Vogl, 2013) and distance to wildlife (mediated by population density)
also affects attitudes (Karlsson \& Sjöström, 2007). Therefore, it would
be valuable to compare how attitudes might vary across different
regions, including developing countries with less influential
conservation movements. The possibility of a defensive `island
mentality'' from the presence of a geographic barrier (such as the
English Channel) may also create more negative attitudes to range-shift
than in more connected regions like North America or mainland Europe.

Our finding that participants were more positive about species with
which they had some experience suggests that familiarity can make it
easier to mobilise support. Public engagement, through recording or
events such as BioBlitz, may therefore be a powerful tool to increase
positive public opinion (Postles \& Bartlett, 2018). We interpreted both
having seen the species and recording the species' taxonomic group as
linking to the same latent construct: familiarity. This is important as
familiarity is unlikely to have a fixed relationship with attitude over
time. For example, as Lynx continue re-establishing in Eastern Europe,
attitudes to them appeared to have worsened as they became more abundant
and more negative impacts appeared (Červený, Krojerová-Prokešová, Kušta,
\& Koubek, 2019). Similarly, changes have occurred in the case of
non-native parakeets establishing in the UK, with some groups hardening
views as impacts emerge and some growing more tolerant as the parakeets
integrate into their sense of place (Crowley, Hinchliffe, \& McDonald,
2019). Future studies will be needed to investigate how attitudes may
change over time and the extent to which familiarity might mediate these
changes to produce complex dynamics. As well as range-shifters, the
number of invasive alien species establishing is forecast to increase
with climate change (Beaury et al., 2020). Wildlife recorders appeared
able to distinguish these two different but related phenomena, but it
may be that attitudes towards the phenomena could interact with each
other. The relationship between attitudes to invasive species and
attitudes to range-shifting species may be an important area of future
research. The effects on attitudes found where the species matched the
recorders' group of interest or had been seen by them might not have
been mediated by familiarity but rather by other intermediate factors,
such as physical proximity, species abundance, recorder behaviour or
positive interactions leading to differing affective relationships
(Lorimer, 2007). However, our first interpretation is supported by the
absence of a spatial pattern in attitudes and the qualitative data's
emphasis on personal experience.

The climate change experimental treatment did not affect respondents'
attitudes. This ties into the thematic analysis (Table S3), where we saw
40 respondents argue that the focus should be on species impacts rather
than cause of arrival, echoing previous research on attitudes towards
invasive species (de Wal et al., 2015). However, it is also possible
that the experimental treatment was ineffective as most respondents
attributed range-shift to climate change, with or without the prompt.
This interpretation is supported by the text responses given to the
control treatment where there were frequent references to climate change
without any prompt (Appendix 2). The significant remaining individual
variation in attitudes in our models hints that the complexity in
predicting responses may be derived from highly personal factors such as
individuals' belief systems.

Disentangling these factors is likely to require a mixed of quantitative
and qualitative approaches. In addition researchers will need to
recognize the subjectivity that they bring to their studies and
implement approaches to account for this in research practices (Brittain
et al., 2020). We suggest that more direct metrics such as risk
perception (Taylor, Dessai, \& Bruine de Bruin, 2014) and views on the
`dynamism vs balance of nature' (Ladle \& Gillson, 2008) may be a
productive avenue in future research exploring individual variation. For
applied regional studies, the local contexts and respondents' sense of
place may also be important (Masterson et al., 2017). Understanding the
different lenses with which people view range-shifting species would
allow bespoke communications to different stakeholders and potentially a
predictive model for potential for conflict (McCleery, 2009).

The metrics we used to categorise respondents' engagement with recording
had little apparent effect on attitudes, though this could be because
small differences were not detectable with our obtained sample size. Our
study focused solely on wildlife recorders and there are likely to be
differences between our findings for this group and the views of other
publics. An important distinction is that wildlife recorders are likely
to be more scientifically aware of nature than the general public
(Figure S4). Therefore, they may be more likely to hold views on
range-shifting species, one way or the other compared to others who have
not previously considered the issue. Even if the latter use and enjoy
the same natural spaces, other public groups may be more likely to draw
from more indirect material when forming their opinions such as media
articles or attitudes to wildlife in general (Brossard and Nisbet,
2007). Wildlife recorders may also be more aware of the ecological roles
of less popular species like wasps and therefore happier to tolerate
arriving range-shifters (Schönfelder and Bogner, 2017). Most wildlife
recorders in our sample aligned with ``stewardship'' in their
relationship with nature and other alignments might indicate different
attitudes towards range-shifters. For example, we could imagine
supporters of ``compassionate conservation'' such as animal rights
activists taking a stronger stand against controlling harmful arrivals,
or against assisted translocation if it were seen to compromise welfare
(Callen et al., 2020; Griffin et al., 2020). Those who derive payment
from ecosystem services such as developers or farmers could seek to
incorporate these species into such schemes such as biodiversity offsets
with ramifications for broader conservation.

The strong relation between attitudes to species and to their management
is intuitive but not inevitable (Lindemann-Matthies, 2016). The
demonstration of this relationship shows that changing views of species
are likely to have knock-on effects on management through changes in
public support. However, our study did not cover all management
scenarios and some information that could have informed respondents'
attitudes towards management was not available, for example estimates of
cost, feasibility, effectiveness, species impact and welfare
implications. It will be important to investigate in the future how
these self-reported attitudes translate when respondents are given more
detailed scenarios, or real case-studies. But there is some evidence
that experimental surveys such as this can align well with real world
behaviour (Hainmueller, Hangartner, \& Yamamoto, 2015).

The predominance of support for non-intervention echoes Ohsawa and Jones
(2017) who found a majority of surveyed park managers preferred not to
intervene at the prospect of species range-shifting through the Japanese
archipelago. However, the support for the non-interventionist management
option is striking as it sits at odds with the typical style of
conservation management in the UK, which is frequently characterised as
interventionist (Adams, 1997). Therefore, the finding that most
respondents expressed a `stewardship' relationship with nature could be
further deconstructed into two more precisely defined `stewardship'-type
relationships. The first more traditional aspect of stewardship is the
archetype of the pragmatist farmer-manager who inventories and actively
supervises nature. The second is a more passive stewardship, protecting
nature as its own agent for future generations. The thematic analysis
suggests that respondents' preferences for non-intervention could be
aligned to both of these aspects of stewardship. Many aligned with
pragmatic stewardship, believing that intervention would be ineffective
and ``there is no point being like King Cnut and trying to hold the tide
back'' and seeing ``no need to throw money into trying to increase
numbers of a naturally increasing species''. Others aligned with more
passive stewardship, emphasising the importance of allowing ``nature''
to choose its own path, espousing ``Nature ebbs and flows, \ldots{} -
that's just how it is'', and ``if we intervine (sic) then it is being
farmed''. The prevalence of passive stewardship ideals, in contrast to
the UK's typical pragmatic style of conservation, could be linked to the
increasing discourse around rewilding and a desire to reduce the
intensity of management (Root-Bernstein, Gooden, \& Boyes, 2018). A need
for wild agency emerges from another comment on range-shifting little
egrets:

\begin{quote}
``No huge sums of money thrown at them, none of this rubbish as per
White Storks at Knepp or Ospreys at Rutland - this was the real deal,
they colonised by themselves''
\end{quote}

Finally, rather than indicating a pragmatic or passive stewardship
perspective a preference, support for non-intervention could represent a
non-committal ``sitting on the fence'' option. This interpretation is
supported by qualitative responses from respondents who indicated that
they felt they lacked necessary information to make decisions on
range-shift management at this time (Sturgis, Roberts, \& Smith, 2014).
Untangling these perspectives and their prevalence will help
conservationists to understand the public mood in their management of
range-shifters.

The lack of support for interventions to support range-shifters could
hinder future attempts to translocate species that are unable to move
fast enough to track their climatic niche. A previous study on assisted
translocation found opposition against moving species outside their
current ranges amongst the British Columbian public (Peterson
St-Laurent, Hagerman, \& Kozak, 2018). In both ours and Peterson's
results intervention to reintroduce locally extinct species were not
opposed. In our study, attitudes to management often favoured native
species over range-shifters when there was a conflict, for example: ``If
it's {[}the range-shifter is{]} having a deleterious effect on native
wildlife then I would support action against it''. We interpret this
attitude as an aspect of a ``balance of nature'' paradigm, where
respondents feel we should protect the natural world from anthropogenic
change (a common belief expressed by our respondents, Table 2). However,
this paradigm contains implicit value judgements often using a fixed
historical baseline as pointed out by another respondent, ``There is an
innate compulsion to resist change, to turn the clock back, to control
and label species as good or bad''. Conservationists may therefore need
to communicate more clearly the alternate paradigm of chaotic, dynamic
nature, which is now relatively widespread in academia (Wu \& Loucks,
1995) but may be less prevalent amongst recorders. Recognition of this
dynamism will be vital to allow range-shifts to protect vulnerable
species from extinction while mitigating the threat to endangered
natives (Scheffers \& Pecl, 2019).

The opposition to measures to remove range-shifters (Figure 3)
superficially suggests that managers may face opposition if they take
such action. However, the text responses elucidate this feeling as being
contingent on the perception that range-shifters pose little threat.
Many respondents were willing to intervene if a threat became apparent.
This focus on demonstrated threat appears in conflict with the
precautionary principle often invoked in invasion biology, i.e.~better
not to introduce taxa just in case there is a risk (Finnoff, Shogren,
Leung, \& Lodge, 2007). We interpret this as a pragmatic response,
demonstrating awareness that efforts to control the propagule pressure
and spread of range-shifting species could be more challenging than for
introduced species (Essl et al 2019). In addition, this reluctance to
intervene suggests that respondents perceive the threat range-shifters
pose to recipient ecosystems is lower than the perceived benefits of
action, and points to the need for urgent research into such threats.
Respondents favoured adapting recipient ecosystems more than controlling
range-shifters themselves, thus they might be more supportive of
management if presented with information on vulnerability of recipient
ecosystems rather than the riskiness of range-shifters.

In summary, we found that wildlife recorders viewed range-shifters more
as vulnerable `ecological refugees' than as threatening `climate
opportunists'. However, recorders were willing to shift their opinions
in response to evidence of harm to native species. In expanding these
results to different stakeholder groups and situations, policy makers
could explore how different policy options might be received under
different future rates of range-shifter arrival and impact. It is
important that any such planning recognises the distinctness of these
new range-shifting species from introduced species and their increasing
relevance for regional conservation (Essl et al 2019). The strong
support for non-interventionist management should be considered by
conservationists when planning management. These attitudes also indicate
a need for stronger scientific advocacy for vulnerable species of
minimal risk -- if, in the future, conserving them requires active
measures including assisted translocation.

\hypertarget{references}{%
\subsection{References}\label{references}}

Adams, W. M. (1997). Rationalization and Conservation: Ecology and the
Management of Nature in the United Kingdom. Transactions of the
Institute of British Geographers, 22(3), 277--291. doi:
10.1111/j.0020-2754.1997.00277.x Andreu, J., Vilà, M., \& Hulme, P. E.
(2009). An Assessment of Stakeholder Perceptions and Management of
Noxious Alien Plants in Spain. Environmental Management, 43(6), 1244.
doi: 10.1007/s00267-009-9280-1 Auspurg, K., \& Jäckle, A. (2017). First
Equals Most Important? Order Effects in Vignette-Based Measurement.
Sociological Methods \& Research, 46(3), 490--539. doi:
10.1177/0049124115591016 Baruch-Mordo, S., Breck, S. W., Wilson, K. R.,
\& Broderick, J. (2009). A Tool Box Half Full: How Social Science can
Help Solve Human--Wildlife Conflict. Human Dimensions of Wildlife,
14(3), 219--223. doi: 10.1080/10871200902839324 Bath, A. J. (1989). The
public and wolf reintroduction in Yellowstone National Park. Society \&
Natural Resources, 2(1), 297--306. doi: 10.1080/08941928909380693
Beaury, E. M., Fusco, E. J., Jackson, M. R., Laginhas, B. B., Morelli,
T. L., Allen, J. M., \ldots{} Bradley, B. A. (2020). Incorporating
climate change into invasive species management: insights from managers.
Biological Invasions, 22(2), 233--252. doi: 10.1007/s10530-019-02087-6
Berry, L. H., Koski, J., Verkuijl, C., Strambo, C., \& Piggot, G.
(2019). Making space: how public participation shapes environmental
decision-making. Stockholm Environment Institute. Borgi, M., \& Cirulli,
F. (2015). Attitudes toward Animals among Kindergarten Children: Species
Preferences. Anthrozoös, 28(1), 45--59. doi:
10.2752/089279315X14129350721939 Bremner, A., \& Park, K. (2007). Public
Attitudes to the Management of Invasive Non-Native Species in Scotland.
Biological Conservation, 139, 306--314. doi:
10.1016/j.biocon.2007.07.005 Brittain, S., Ibbett, H., Lange, E. de,
Dorward, L., Hoyte, S., Marino, A., Milner‐Gulland, E.J., Newth, J.,
Rakotonarivo, S., Veríssimo, D., Lewis, J., 2020. Ethical considerations
when conservation research involves people. Conservation Biology 34,
925--933. \url{https://doi.org/10.1111/cobi.13464} Brooks, S. P., \&
Gelman, A. (1998). General methods for monitoring convergence of
iterative simulations. Journal of Computational and Graphical
Statistics, 7(4), 434--455. doi: 10.2307/1390675 Brossard, D., Nisbet,
M.C., 2007. Deference to Scientific Authority Among a Low Information
Public: Understanding U.S. Opinion on Agricultural Biotechnology.
International Journal of Public Opinion Research 19, 24--52.
\url{https://doi.org/10.1093/ijpor/edl003} Brown, P. M. J., Roy, D. B.,
Harrower, C., Dean, H. J., Rorke, S. L., \& Roy, H. E. (2018). Spread of
a model invasive alien species, the harlequin ladybird Harmonia axyridis
in Britain and Ireland. Scientific Data, 5(1), 180239. doi:
10.1038/sdata.2018.239 Bürkner, P.-C. (2018). Advanced Bayesian
Multilevel Modeling with the R Package brms. The R Journal, 10(1),
395--411. doi: 10.32614/RJ-2018-017 Callen, A., Hayward, M.W.,
Klop-Toker, K., Allen, B.L., Ballard, G., Beranek, C.T., Broekhuis, F.,
Bugir, C.K., Clarke, R.H., Clulow, J., Clulow, S., Daltry, J.C.,
Davies-Mostert, H.T., Di Blanco, Y.E., Dixon, V., Fleming, P.J.S.,
Howell, L.G., Kerley, G.I.H., Legge, S.M., Lenga, D.J., Major, T.,
Montgomery, R.A., Moseby, K., Meyer, N., Parker, D.M., Périquet, S.,
Read, J., Scanlon, R.J., Shuttleworth, C., Tamessar, C.T., Taylor, W.A.,
Tuft, K., Upton, R.M.O., Valenzuela, M., Witt, R.R., Wüster, W., 2020.
Envisioning the future with `compassionate conservation': An ominous
projection for native wildlife and biodiversity. Biological Conservation
241, 108365. \url{https://doi.org/10.1016/j.biocon.2019.108365} Červený,
J., Krojerová-Prokešová, J., Kušta, T., \& Koubek, P. (2019). The change
in the attitudes of Czech hunters towards Eurasian lynx: Is poaching
restricting lynx population growth? Journal for Nature Conservation, 47,
28--37. doi: 10.1016/j.jnc.2018.11.002 Chen, I.-C., Hill, J.,
Ohlemüller, R., Roy, D.B., Thomas, C., 2011. Rapid Range Shifts of
Species Associated with High Levels of Climate Warming. Science (New
York, N.Y.) 333, 1024--6. \url{https://doi.org/10.1126/science.1206432}
Crowley, S. L., Hinchliffe, S., \& McDonald, R. A. (2017). Conflict in
invasive species management. Frontiers in Ecology and the Environment,
15(3), 133--141. doi: 10.1002/fee.1471 Crowley, S. L., Hinchliffe, S.,
\& McDonald, R. A. (2019). The parakeet protectors: Understanding
opposition to introduced species management. Journal of Environmental
Management, 229, 120--132. doi: 10.1016/j.jenvman.2017.11.036 Dawson,
N., \& Martin, A. (2015). Assessing the contribution of ecosystem
services to human wellbeing: A disaggregated study in western Rwanda.
Ecological Economics, 117, 62--72. doi: 10.1016/j.ecolecon.2015.06.018
de Groot, M., Drenthen, M., \& de Groot, W. T. (2011). Public Visions of
the Human/Nature Relationship and their Implications for Environmental
Ethics: Environmental Ethics, 33(1), 25--44. doi:
10.5840/enviroethics20113314 de Groot, W. T., \& van den Born, R. J. G.
(2003). Visions of nature and landscape type preferences: an exploration
in The Netherlands. Landscape and Urban Planning, 63(3), 127--138. doi:
10.1016/S0169-2046(02)00184-6 Essl, F., Dullinger, S., Genovesi, P.,
Hulme, P.E., Jeschke, J.M., Katsanevakis, S., Kühn, I., Lenzner, B.,
Pauchard, A., Pyšek, P., Rabitsch, W., Richardson, D.M., Seebens, H.,
van Kleunen, M., van der Putten, W.H., Vilà, M., Bacher, S., 2019. A
Conceptual Framework for Range-Expanding Species that Track
Human-Induced Environmental Change. BioScience 69, 908--919.
\url{https://doi.org/10.1093/biosci/biz101} Finnoff, D., Shogren, J. F.,
Leung, B., \& Lodge, D. (2007). Take a risk: Preferring prevention over
control of biological invaders. Ecological Economics, 62(2), 216--222.
doi: 10.1016/j.ecolecon.2006.03.025 Franzen, A., \& Vogl, D. (2013). Two
decades of measuring environmental attitudes: A comparative analysis of
33 countries. Global Environmental Change, 23(5), 1001--1008. doi:
10.1016/j.gloenvcha.2013.03.009 Griffin, A.S., Callen, A., Klop-Toker,
K., Scanlon, R.J., Hayward, M.W., 2020. Compassionate Conservation
Clashes With Conservation Biology: Should Empathy, Compassion, and
Deontological Moral Principles Drive Conservation Practice? Front.
Psychol. 11. \url{https://doi.org/10.3389/fpsyg.2020.01139} Gurney, M.
(2015). Gains and losses: extinctions and colonisations in Britain since
1900. Biological Journal of the Linnean Society, 115(3), 573--585. doi:
10.1111/bij.12503 Hainmueller, J., Hangartner, D., \& Yamamoto, T.
(2015). Validating vignette and conjoint survey experiments against
real-world behavior. Proceedings of the National Academy of Sciences,
112(8), 2395--2400. doi: 10.1073/pnas.1416587112 Jaric, I., Courchamp,
F., Correia, R. A., Crowley, S. L., Essl, F., Fischer, A., \ldots{}
Jeschke, J. M. (2020). The role of species charisma in biological
invasions. Frontiers in Ecology and the Environment, 18(6), 345--352.
doi: 10.1002/fee.2195 Karlsson, J., \& Sjöström, M. (2007). Human
attitudes towards wolves, a matter of distance. Biological Conservation,
137(4), 610--616. doi: 10.1016/j.biocon.2007.03.023 Kiss, G. (2014). Why
Should the Public Participate in Environmental Decision-Making?
Theoretical Arguments for Public Participation. Periodica Polytechnica
Social and Management Sciences, 22(1), 13--20. doi: 10.3311/PPso.7400
Klich, D., Olech, W., Łopucki, R., \& Danik, K. (2018). Community
attitudes to the European bison Bison bonasus in areas where its
reintroduction is planned and in areas with existing populations in
northeastern Poland. European Journal of Wildlife Research, 64(5), 61.
doi: 10.1007/s10344-018-1219-5 Jones, L.P., Turvey, S.T., Massimino, D.,
Papworth, S.K. (2020). Investigating the implications of shifting
baseline syndrome on conservation. People and Nature 2, 1131--1144.
\url{https://doi.org/10.1002/pan3.10140} Ladle, R. J., \& Gillson, L.
(2008). The (im)balance of nature: a public perception time-lag?: Public
Understanding of Science. (Sage UK: London, England). doi:
10.1177/0963662507082893 Latombe, G., Essl, F., McGeoch, M.A., 2020. The
effect of cross-boundary management on the trajectory to commonness in
biological invasions. NeoBiota 62, 241--267.
\url{https://doi.org/10.3897/neobiota.62.52708} Lenoir, J., Svenning,
J.-C., 2015. Climate-related range shifts -- a global multidimensional
synthesis and new research directions. Ecography 38, 15--28.
\url{https://doi.org/10.1111/ecog.00967} Lindemann-Matthies, P. (2016).
Beasts or beauties? Laypersons' perception of invasive alien plant
species in Switzerland and attitudes towards their management. NeoBiota,
29, 15--33. doi: 10.3897/neobiota.29.5786 Limesurvey GmbH. / LimeSurvey:
An Open Source survey tool /LimeSurvey GmbH, Hamburg, Germany. URL
\url{http://www.limesurvey.org} Lorimer, J. (2007). Nonhuman Charisma.
Environment and Planning D: Society and Space, 25(5), 911--932. doi:
10.1068/d71j Lunt, I. D., Byrne, M., Hellmann, J. J., Mitchell, N. J.,
Garnett, S. T., Hayward, M. W., \ldots{} Zander, K. K. (2013). Using
assisted colonisation to conserve biodiversity and restore ecosystem
function under climate change. Biological Conservation, 157, 172--177.
doi: 10.1016/j.biocon.2012.08.034 Ma, K., Liu, D., Wei, R., Zhang, G.,
Xie, H., Huang, Y., \ldots{} Xu, H. (2016). Giant panda reintroduction:
factors affecting public support. Biodiversity and Conservation, 25(14),
2987--3004. doi: 10.1007/s10531-016-1215-6 Masterson, V. A., Stedman, R.
C., Enqvist, J., Tengö, M., Giusti, M., Wahl, D., \& Svedin, U. (2017).
The contribution of sense of place to social-ecological systems
research: a review and research agenda. Ecology and Society, 22(1).
JSTOR. Retrieved from \url{https://www.jstor.org/stable/26270120}
McCleery, R. A. (2009). Improving Attitudinal Frameworks to Predict
Behaviors in Human--Wildlife Conflicts. Society \& Natural Resources,
22(4), 353--368. doi: 10.1080/08941920802064414 Naujokaitis-Lewis, I.,
Pomara, L. Y., \& Zuckerberg, B. (2018). Delaying conservation actions
matters for species vulnerable to climate change. Journal of Applied
Ecology, 55(6), 2843--2853. doi: 10.1111/1365-2664.13241 Newbold, T.
(2018). Future effects of climate and land-use change on terrestrial
vertebrate community diversity under different scenarios. Proceedings of
the Royal Society B: Biological Sciences, 285(1881), 20180792. doi:
10.1098/rspb.2018.0792 Ohsawa, T., \& Jones, T. E. (2017). How Can
Protected Area Managers Deal with Nonnative Species in an Era of Climate
Change? Natural Areas Journal, 37(2), 240--253. doi:
10.3375/043.037.0213 O'Rourke, E. (2014). The reintroduction of the
white-tailed sea eagle to Ireland: People and wildlife. Land Use Policy,
38, 129--137. doi: 10.1016/j.landusepol.2013.10.020 Pecl, G. T., Araújo,
M. B., Bell, J. D., Blanchard, J., Bonebrake, T. C., Chen, I.-C.,
\ldots{} Williams, S. E. (2017). Biodiversity redistribution under
climate change: Impacts on ecosystems and human well-being. Science,
355(6332). doi: 10.1126/science.aai9214 Peterson St-Laurent, G.,
Hagerman, S., \& Kozak, R. (2018). What risks matter? Public views about
assisted migration and other climate-adaptive reforestation strategies.
Climatic Change, 151(3), 573--587. doi: 10.1007/s10584-018-2310-3
Piironen, J., \& Vehtari, A. (2017). Sparsity information and
regularization in the horseshoe and other shrinkage priors. Electronic
Journal of Statistics, 11(2), 5018--5051. doi: 10.1214/17-EJS1337SI
Pocock, M. J. O., Roy, H. E., Preston, C. D., \& Roy, D. B. (2015). The
Biological Records Centre: a pioneer of citizen science. Biological
Journal of the Linnean Society, 115(3), 475--493. doi: 10.1111/bij.12548
Postles, M., \& Bartlett, M. (2018). The rise of BioBlitz: Evaluating a
popular event format for public engagement and wildlife recording in the
United Kingdom. Applied Environmental Education \& Communication, 17(4),
365--379. doi: 10.1080/1533015X.2018.1427010 QSR International. (1999).
NVivo Qualitative Data Analysis Software (Version NVivo 12). Retrieved
from Available from
\url{https://qsrinternational.com/nvivo/nvivo-products/} R Core Team.
(2020). R: A language and environment for statistical computing.
(Version 4.0.2). Vienna, Austria.: R Foundation for Statistical
Computing. Retrieved from URL \url{https://www.R-project.org/}. Redpath,
S. M., Young, J., Evely, A., Adams, W. M., Sutherland, W. J.,
Whitehouse, A., \ldots{} Gutiérrez, R. J. (2013). Understanding and
managing conservation conflicts. Trends in Ecology \& Evolution, 28(2),
100--109. doi: 10.1016/j.tree.2012.08.021 Root-Bernstein, M., Gooden,
J., \& Boyes, A. (2018). Rewilding in practice: Projects and policy.
Geoforum, 97, 292--304. doi: 10.1016/j.geoforum.2018.09.017 Rouanet, H.,
\& Le Roux, B. (2010). The Method of Multiple Correspondence Analysis.
In Quantitative Applications in the Social Sciences: Multiple
correspondence analysis. Thousand Oaks, California: SAGE Publications,
Inc.~doi: 10.4135/9781412993906 Scheffers, B. R., \& Pecl, G. (2019).
Persecuting, protecting or ignoring biodiversity under climate change.
Nature Climate Change, 9(8), 581--586. doi: 10.1038/s41558-019-0526-5
Schönfelder, M.L., Bogner, F.X., 2017. Individual perception of bees:
Between perceived danger and willingness to protect. PLOS ONE 12,
e0180168. \url{https://doi.org/10.1371/journal.pone.0180168} Seddon, P.
J., Soorae, P. S., \& Launay, F. (2005). Taxonomic bias in
reintroduction projects. Animal Conservation, 8(1), 51--58. doi:
10.1017/S1367943004001799 Sharp, R. L., Larson, L. R., \& Green, G. T.
(2011). Factors influencing public preferences for invasive alien
species management. Biological Conservation, 144(8), 2097--2104. doi:
10.1016/j.biocon.2011.04.032 Shipley, N. J., \& Bixler, R. D. (2017).
Beautiful Bugs, Bothersome Bugs, and FUN Bugs: Examining Human
Interactions with Insects and Other Arthropods. Anthrozoös, 30(3),
357--372. doi: 10.1080/08927936.2017.1335083 Sturgis, P., Roberts, C.,
\& Smith, P. (2014). Middle Alternatives Revisited: How the neither/nor
Response Acts as a Way of Saying ``I Don't Know''? Sociological Methods
\& Research, 43(1), 15--38. doi: 10.1177/0049124112452527 Sumner, S.,
Law, G., \& Cini, A. (2018). Why we love bees and hate wasps. Ecological
Entomology, 43(6), 836--845. doi: 10.1111/een.12676 Taylor, A. L.,
Dessai, S., \& Bruine de Bruin, W. (2014). Public perception of climate
risk and adaptation in the UK: A review of the literature. Climate Risk
Management, 4--5, 1--16. doi: 10.1016/j.crm.2014.09.001 Tebboth, M. G.
L., Few, R., Assen, M., \& Degefu, M. A. (2020). Valuing local
perspectives on invasive species management: Moving beyond the ecosystem
service-disservice dichotomy. Ecosystem Services, 42, 101068. doi:
10.1016/j.ecoser.2020.101068 Troudet, J., Grandcolas, P., Blin, A.,
Vignes-Lebbe, R., \& Legendre, F. (2017). Taxonomic bias in biodiversity
data and societal preferences. Scientific Reports, 7(1), 9132. doi:
10.1038/s41598-017-09084-6 Urban, M.C., 2020. Climate-tracking species
are not invasive. Nature Climate Change 10, 382--384.
\url{https://doi.org/10.1038/s41558-020-0770-8} Verbrugge, L., Born, R.,
\& Lenders, H. (2013). Exploring public perception of non-native species
from a visions of nature perspective. Environmental Management, 52. doi:
10.1007/s00267-013-0170-1 Van der Wal, R. , Fischer, A., Selge, S., \&
Larson, B. M. H. (2015). Neither the public nor experts judge species
primarily on their origins. Environmental Conservation, 42(4), 349--355.
doi: 10.1017/S0376892915000053 Wallingford, P.D., Morelli, T.L., Allen,
J.M., Beaury, E.M., Blumenthal, D.M., Bradley, B.A., Dukes, J.S., Early,
R., Fusco, E.J., Goldberg, D.E., Ibanez, I., Laginhas, B.B., Vila, M.,
Sorte, C.J.B., 2020. Adjusting the lens of invasion biology to focus on
the impacts of climate-driven range shifts. Nat. Clim. Chang. 10,
398--405. \url{https://doi.org/10.1038/s41558-020-0768-2} Wu, J., \&
Loucks, O. L. (1995). From Balance of Nature to Hierarchical Patch
Dynamics: A Paradigm Shift in Ecology. The Quarterly Review of Biology,
70(4), 439--466. doi: 10.1086/419172 Limesurvey GmbH. / LimeSurvey: An
Open Source survey tool /LimeSurvey GmbH, Hamburg, Germany. URL
\url{http://www.limesurvey.org}

\hypertarget{acknowledgements}{%
\subsection{Acknowledgements:}\label{acknowledgements}}

I would like to acknowledge the support of my NERC GW4+ Doctoral
Training Partnership studentship from the Natural Environment Research
Council NERC grant NE/N008669/1. The wildlife recorders who responded to
and helped disseminate the survey. The National Biodiversity Network for
and Henry Häkkinen for his help with using CSS in the survey design.

\hypertarget{conflicts-of-interest}{%
\subsection{Conflicts of Interest:}\label{conflicts-of-interest}}

None Known

\hypertarget{authors-contributions}{%
\subsection{Authors' Contributions:}\label{authors-contributions}}

JC and RE conceived the scope and questions of the study. SC supported
the development of the methodology and refinement of the research
questions. JC designed the survey instrument with feedback from RE and
SC. JC distributed the survey and collected responses. JC led the
analysis and interpretation of the data with the input of RE and SC. All
authors contributed critically to the drafts and gave final approval for
publication.

\hypertarget{data-availability-statement}{%
\subsection{Data Availability
Statement:}\label{data-availability-statement}}

We intend to make the survey design and responses available on the dryad
data repository. Code for the analysis shall be made available in a
public GitHub repository.

\hypertarget{orcids}{%
\subsection{ORCIDs:}\label{orcids}}

James Cranston \url{https://orcid.org/0000-0002-1009-4763}\\
Sarah L. Crowley \url{https://orcid.org/0000-0002-4854-0925}\\
Regan Early \url{https://orcid.org/0000-0003-4108-5904}

\end{document}
